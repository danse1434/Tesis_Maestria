%----------------------------------------------------------------------------------------
%	VARIOS PAQUETES Y CONFIGURACIONES REQUERIDAS
%----------------------------------------------------------------------------------------

% Configuración de lenguaje
\usepackage[spanish,es-nodecimaldot, es-tabla]{babel} % Lenguaje Español, uso de punto en vez e coma para decimales, uso de tabla
\usepackage{fancyhdr}
\usepackage{epsfig}
\usepackage{epic}
\usepackage{eepic}
\usepackage{amsmath}
\usepackage{threeparttable}
\usepackage{amscd}
\usepackage{here}
\usepackage{graphicx}
\usepackage{lscape}
\usepackage{tabularx}
\usepackage{subfigure}
\usepackage{longtable}


\usepackage{rotating} %Para rotar texto, objetos y tablas seite. No se ve en DVI solo en PS. Seite 328 Hundebuch
                        %se usa junto con \rotate, \sidewidestable ....


%----------------------------------------------------------------------------------------
%	Fuentes
%----------------------------------------------------------------------------------------

\usepackage{avant} % Use the Avantgarde font for headings
%\usepackage{times} % Use the Times font for headings

% Usar fuente arial de manera predeterminada
\usepackage{helvet}
\renewcommand{\familydefault}{\sfdefault}

% Cambia los casos donde aparece Mathcal
\DeclareMathAlphabet{\mathcal}{OMS}{cmsy}{m}{n}


\usepackage{microtype} % Slightly tweak font spacing for aesthetics
\usepackage[utf8]{inputenc} % Required for including letters with accents
\usepackage[T1]{fontenc} % Use 8-bit encoding that has 256 glyphs
\usepackage{subfig}

%----------------------------------------------------------------------------------------
%	Bibliografía e Índice
%----------------------------------------------------------------------------------------

%\usepackage{csquotes}
%	\usepackage[style=numeric,citestyle=numeric,sorting=nyt,sortcites=true,autopunct=true,autolang=hyphen,hyperref=true,abbreviate=false,backref=true,backend=biber,defernumbers=true]{biblatex}
%\addbibresource{bibliography.bib} % BibTeX bibliography file
%\defbibheading{bibempty}{}

\usepackage[numbers, round, sort, sectionbib]{natbib}
\renewcommand{\bibsection}{} % Para quitar la palabra bibliografía como título de esta sección o colocar otra. 
\setcitestyle{square}
\bibliographystyle{unsrtnat}


\usepackage{calc} % For simpler calculation - used for spacing the index letter headings correctly
\usepackage{makeidx} % Required to make an index
\makeindex % Tells LaTeX to create the files required for indexing



%----------------------------------------------------------------------------------------
%	
%----------------------------------------------------------------------------------------

\renewcommand{\theequation}{\thechapter-\arabic{equation}}
\renewcommand{\thefigure}{\textbf{\thechapter-\arabic{figure}}}
\renewcommand{\thetable}{\textbf{\thechapter-\arabic{table}}}


%\pagestyle{fancyplain} %\addtolength{\headwidth}{\marginparwidth}
%\textheight22.5cm \topmargin0cm \textwidth16.5cm
%\oddsidemargin0.25cm \evensidemargin-0.25cm  % \oddsidemargin0.5cm \evensidemargin-0.5cm%

\usepackage[
top=2.54cm,
bottom=2.54cm,
left=3.6cm,
right=2.54cm,
headheight=17pt, % as per the warning by fancyhdr
includehead,includefoot,
heightrounded, % to avoid spurious underfull messages
]{geometry} 

\renewcommand{\chaptermark}[1]{\markboth{\thechapter\; #1}{}}
\renewcommand{\sectionmark}[1]{\markright{\thesection\; #1}}
\lhead[\fancyplain{}{\thepage}]{\fancyplain{}{\rightmark}}
\rhead[\fancyplain{}{\leftmark}]{\fancyplain{}{\thepage}}
\fancyfoot{}
\thispagestyle{fancy}%


\addtolength{\headwidth}{0cm}
\unitlength1mm %Define la unidad LE para Figuras
\mathindent0cm %Define la distancia de las formulas al texto,  fleqn las descentra
\marginparwidth0cm
\parindent0cm %Define la distancia de la primera linea de un parrafo a la margen

%Para tablas,  redefine el backschlash en tablas donde se define la posici\'{o}n del texto en las
%casillas (con \centering \raggedright o \raggedleft)
\newcommand{\PreserveBackslash}[1]{\let\temp=\\#1\let\\=\temp}
\let\PBS=\PreserveBackslash

%Espacio entre lineas
\renewcommand{\baselinestretch}{1.1}

%Neuer Befehl f\"{u}r die Tabelle Eigenschaften der Aktivkohlen
\newcommand{\arr}[1]{\raisebox{1.5ex}[0cm][0cm]{#1}}

%Neue Kommandos
\usepackage{Befehle}
%Inicio del documento. Tener en cuenta que hay archivos auxiliares
