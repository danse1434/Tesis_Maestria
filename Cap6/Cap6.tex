\chapter{Conclusiones y recomendaciones}
\section{Conclusiones}
Las conclusiones constituyen un cap\'{\i}tulo independiente y presentan, en forma l\'{o}gica, los resultados de la tesis  o trabajo de investigaci\'{o}n. Las conclusiones deben ser la respuesta a los objetivos o prop\'{o}sitos planteados. Se deben titular con la palabra conclusiones en el mismo formato de los t\'{\i}tulos de los cap\'{\i}tulos anteriores (T\'{\i}tulos primer nivel), precedida por el numeral correspondiente (seg\'{u}n la presente plantilla).\\

\section{Recomendaciones}
Se presentan como una serie de aspectos que se podr\'{\i}an realizar en un futuro para emprender investigaciones similares o fortalecer la investigaci\'{o}n realizada. Deben contemplar las perspectivas de la investigaci\'{o}n, las cuales son sugerencias, proyecciones o alternativas que se presentan para modificar, cambiar o incidir sobre una situaci\'{o}n espec\'{\i}fica o una problem\'{a}tica encontrada. Pueden presentarse como un texto con caracter\'{\i}sticas argumentativas, resultado de una reflexi\'{o}n acerca de la tesis o trabajo de investigaci\'{o}n.\\